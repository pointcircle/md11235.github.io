% Created 2017-11-29 周三 13:03
% Intended LaTeX compiler: pdflatex

\documentclass[12pt,a4paper]{article}
\def\pgfsysdriver{pgfsys-dvipdfm.def}

\usepackage{fontspec,indentfirst}
\usepackage{xunicode}% provides unicode character macros
\usepackage{xltxtra} % provides some fixes/extras
\usepackage[margin=1.5cm]{geometry}
\usepackage{tikz}

\XeTeXlinebreaklocale "zh"
\XeTeXlinebreakskip = 0pt plus 1pt minus 0.1pt


\newfontfamily\xingkai{"XHei-Mono+AnonymousPro"}
\newfontfamily\caiyun{"XHei-Mono+AnonymousPro"}
\newfontfamily\kai{"XHei-Mono+AnonymousPro"}
\newfontfamily\fs{"XHei-Mono+AnonymousPro"}
\newfontfamily\li{"XHei-Mono+AnonymousPro"}
\newfontfamily\xinwei{"XHei-Mono+AnonymousPro"}
\newfontfamily\yao{"XHei-Mono+AnonymousPro"}
\newfontfamily\hei{"XHei-Mono+AnonymousPro"}
\newfontfamily\song{"XHei-Mono+AnonymousPro"}

\setmainfont{"XHei-Mono+AnonymousPro"}

\renewcommand{\baselinestretch}{1.2}

% [FIXME] ox-latex 的設計不良導致 hypersetup 必須在這裡插入
\usepackage{hyperref}
\hypersetup{
  colorlinks=true, %把紅框框移掉改用字體顏色不同來顯示連結
  linkcolor=[rgb]{0,0.37,0.53},
  citecolor=[rgb]{0,0.47,0.68},
  filecolor=[rgb]{0,0.37,0.53},
  urlcolor=[rgb]{0,0.37,0.53},
  pagebackref=true,
  linktoc=all,}
\author{md11235}
\date{\today}
\title{如何利用VNC访问平时没有图形界面的服务器}
\begin{document}

\maketitle
\tableofcontents

比如有的时候,需要用图形界面跑跑虚拟机什么的。

\section{安装桌面环境、vncserver等}
\label{sec:orga1a0543}

轻量些的环境比如Xfce4。以ubuntu服务器为例:

\begin{verbatim}
sudo apt install xfce4 xfce4-goodies tightvncserver xrdp
\end{verbatim}

\section{配置vncserver}
\label{sec:orge6fbdd9}

建立\textasciitilde{}/.vnc目录,设置vnc连接的密码:
\begin{verbatim}
mkdir -p ~/.vnc
echo password | vncpasswd -f > ~/.vnc/passwd
chmod 600 ~/.vnc/passwd
\end{verbatim}

然后写初始化文件xstartup:

\begin{verbatim}
#!/bin/bash
xrdb $HOME/.Xresources
startxfce4 &
\end{verbatim}

再给它加上x权限:
\begin{verbatim}
chmod +x ~/.vnc/xstartup
\end{verbatim}

\section{运行vncserver}
\label{sec:orgb79a32f}

用pietty等工具ssh到服务器以后,根据具体需要的分辨率运行vncserver:

\begin{verbatim}
vncserver -geometry 1920x1808
\end{verbatim}

这个命令会输出类似信息:

\begin{quote}
New 'X' desktop is your-host-name:2
\end{quote}

\section{用VNC客户端连接到服务器}
\label{sec:org0d9eb8b}

假使服务器的IP地址是10.0.0.3。在Windows上运行vnc viewer,比如
VNC-Viewer-5.3.0-Windows-64bit.exe,对其进行如下配置:

在VNC Server一栏输入(注意:2和vncserver的输出里的:2的对应关系:10.0.0.3:2

在Encryption一栏选择“Let VNC Server choose”就行。

然后点击Connect按钮,输入密码,能看到远端的桌面了。
\end{document}